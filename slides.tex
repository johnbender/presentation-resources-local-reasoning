% Copyright 2004 by Till Tantau <tantau@users.sourceforge.net>.
%
% In principle, this file can be redistributed and/or modified under
% the terms of the GNU Public License, version 2.
%
% However, this file is supposed to be a template to be modified
% for your own needs. For this reason, if you use this file as a
% template and not specifically distribute it as part of a another
% package/program, I grant the extra permission to freely copy and
% modify this file as you see fit and even to delete this copyright
% notice.

\documentclass{beamer}
\usefonttheme{serif}
\usepackage{wrapfig}
\usepackage{palatino}
\usepackage{latexsym}
\usepackage{verbatim}
\usepackage{alltt}
\usepackage{amsmath,proof,amsthm,amssymb,enumerate}
\usepackage{mathtools}
\usepackage{mdframed}
\newcommand\hmmax{0}
\newcommand\bmmax{0}
\usepackage{bm}

\newcommand{\mtt}[1]{
  \mathtt{#1}\;
}

\renewcommand{\syntleft}{\normalfont\itshape}
\renewcommand{\syntright}{}

\newcommand{\bfdefault}{bx}

% There are many different themes available for Beamer. A comprehensive
% list with examples is given here:
% http://deic.uab.es/~iblanes/beamer_gallery/index_by_theme.html
% You can uncomment the themes below if you would like to use a different
% one:
%\usetheme{AnnArbor}
%\usetheme{Antibes}
%\usetheme{Bergen}
%\usetheme{Berkeley}
%\usetheme{Berlin}
%\usetheme{Boadilla}
%\usetheme{boxes}
%\usetheme{CambridgeUS}
%\usetheme{Copenhagen}
%\usetheme{Darmstadt}
\usetheme{default}
%\usetheme{Frankfurt}
%\usetheme{Goettingen}
%\usetheme{Hannover}
%\usetheme{Ilmenau}
%\usetheme{JuanLesPins}
%\usetheme{Luebeck}
%\usetheme{Madrid}
%\usetheme{Malmoe}
%\usetheme{Marburg}
%\usetheme{Montpellier}
%\usetheme{PaloAlto}
%\usetheme{Pittsburgh}
%\usetheme{Rochester}
%\usetheme{Singapore}
%\usetheme{Szeged}
%\usetheme{Warsaw}

\title{Presentation Title}

% A subtitle is optional and this may be deleted
\subtitle{Optional Subtitle}

\author{F.~Author\inst{1} \and S.~Another\inst{2}}
% - Give the names in the same order as the appear in the paper.
% - Use the \inst{?} command only if the authors have different
%   affiliation.

\institute[Universities of Somewhere and Elsewhere] % (optional, but mostly needed)
{
  \inst{1}%
  Department of Computer Science\\
  University of Somewhere
  \and
  \inst{2}%
  Department of Theoretical Philosophy\\
  University of Elsewhere}
% - Use the \inst command only if there are several affiliations.
% - Keep it simple, no one is interested in your street address.

\date{Conference Name, 2013}
% - Either use conference name or its abbreviation.
% - Not really informative to the audience, more for people (including
%   yourself) who are reading the slides online

\subject{Theoretical Computer Science}
% This is only inserted into the PDF information catalog. Can be left
% out.

% If you have a file called "university-logo-filename.xxx", where xxx
% is a graphic format that can be processed by latex or pdflatex,
% resp., then you can add a logo as follows:

% \pgfdeclareimage[height=0.5cm]{university-logo}{university-logo-filename}
% \logo{\pgfuseimage{university-logo}}

% Delete this, if you do not want the table of contents to pop up at
% the beginning of each subsection:
\AtBeginSubsection[]
{
  \begin{frame}<beamer>{Outline}
    \tableofcontents[currentsection,currentsubsection]
  \end{frame}
}


% Let's get started
\begin{document}

\setlength{\abovedisplayskip}{0pt}
\setlength{\belowdisplayskip}{0pt}
\setlength{\abovedisplayshortskip}{0pt}
\setlength{\belowdisplayshortskip}{0pt}

\begin{frame}{Separation Logic Review}
  \begin{block}{Pointers and pointer arithmetic}
    $\mapsto$ is the singleton heap. A mapping from an address (Int) to a value contained at that address in the heap.
  \end{block}

  \begin{example}
    \begin{align*}
      10 &\mapsto a \\
      x &\mapsto a,b \\
      x &\mapsto -
    \end{align*}

\vspace{0.5cm}

    \begin{mdframed}[outerlinewidth=2]
      \begin{align*}
        E \mapsto F_0,...F_n & \triangleq (E \mapsto F_0) * ... * (E + n \mapsto F_n)\\
        E \mapsto - & \triangleq \exists F.E \mapsto F \qquad (F \notin Free(E))
      \end{align*}
    \end{mdframed}
  \end{example}
\end{frame}


\begin{frame}{Separation Logic Review}
  \begin{block}{Separating Conjunction}
    If $P$ and $Q$ are true in their own heaps then $P * Q$ is true in the conjunction of those two heaps.
  \end{block}

  \begin{example}
    \begin{align*}
      \vspace*{-1cm}
      (x \mapsto 2) * (x \mapsto 1)
    \end{align*}
  \end{example}

  \begin{block}{Empty Heap}
    $\mtt{emp}$ represents a heap without allocations.
  \end{block}

  \begin{example}
    \begin{align*}
      P \Leftrightarrow\ &\mtt{emp} * P
    \end{align*}
  \end{example}
\end{frame}

\begin{frame}{Race-free Daring Programs}
  \begin{block}{Mutual Exclusion Groups}
    A class of commands where the elements are required not to overlap in execution.
  \end{block}

  \begin{block}{Racy vs Race-free}
    A racy program attempts to access the same portion of state at the same time from different threads of execution.
  \end{block}

  \begin{block}{Daring vs Cautious}
    A daring program accesses shared state outside of mutual exclusion groups.
  \end{block}
\end{frame}

\begin{frame}{Race-free Daring Programs}
  \begin{block}{Split Binary Semaphore}
    The heap mutation and variable assignment happen outside of a mutual exclusion group, but this program is not racy.
  \end{block}

  \begin{example}
    \begin{alignat*}{3}
      & \mathtt{semaphore}\; free := 1;\; &&busy := 0 \\
      & \vdots            && \vdots \\
      & \mathtt{P}(free); && \mathtt{P}(busy); \\
      & [10] := m; \qquad \|     && n := [10]; \\
      & \mathtt{V}(busy); && \mathtt{V}(free); \\
      & \vdots            && \vdots
    \end{alignat*}
  \end{example}
\end{frame}

\begin{frame}{Resources}
  \begin{block}{Resources and Variables}
    \begin{itemize}
    \item{
        A variable belongs to at most one resource.
    }

    \item{
        If a variable is owned by a resource it can only appear in a CCR for that resource.
    }

    \item{
        If a variable is changed in one process it cannot appear in another unless it belongs to a resource.
    }
    \end{itemize}
  \end{block}
  \begin{example}
    \begin{align*}
      & init;\\
      & \mtt{resource} r_1\mathrm{(variable\ list)},\ldots,r_n\mathrm{(variable\ list)} \\
      & \mathit{C_1}\ \| \cdots \|\ \mathit{C_n}\\
    \end{align*}
  \end{example}

\end{frame}


\begin{frame}{Resources}
  \begin{block}{Conditional Critical Regions}
    Unit of mutual exclusion for resources. No other region for $r$ can be executing and $B$ must be true.
  \end{block}

  \begin{example}
    \begin{align*}
      \; &\mtt{with} r\ \mtt{when} B\ \mtt{do} C\ \mtt{endwith}\\
      \mathtt{P}(s)\; \triangleq\; &\mtt{with} s\ \mtt{when} s > 0\ \mtt{do} s := s + 1\ \mtt{endwith}\\
      \mathtt{V}(s)\; \triangleq\; &\mtt{with} s\ \mtt{when} \mtt{true}\ \mtt{do} s := s + 1\ \mtt{endwith}\\
    \end{align*}
  \end{example}
\end{frame}


\begin{frame}{Resources}
  \begin{block}{Resource Invariants}
    Predicates used to establish the ownership of heap sections through reasoning about resource state.
  \end{block}
\end{frame}

\begin{frame}{Expression and Assertion Syntax}
  \begin{itemize}
    \item {
      Syntax

      \begin{alignat*}{1}
        E, F &::=\ x, y, ...\ |\ 0\ |\ 1 |\ E + F\ |\ E \times F\ |\ E - F \\
        B &::=\ \mtt{false}\ |\ B \Rightarrow B\ |\ E = F\ |\ E < F \\
        P, Q, R &::=\ B\ |\ \mtt{emp}\ |\ E \mapsto F\ |\ P * Q\ |\ P \Rightarrow Q\ |\ \forall x.P\ |\ ...
      \end{alignat*}
    }

    \item {
      Abbreviations

      \begin{alignat*}{1}
        \neg P &= P \Rightarrow false\\
        true &\triangleq \neg(false)\\
        P \lor Q &\triangleq (\neg P) \Rightarrow Q\\
        P \land Q &\triangleq \neg (\neg P \lor \neg Q)\\
        \exists x. P &\triangleq \neg \forall x.\neg P \\
      \end{alignat*}
    }
  \end{itemize}
\end{frame}

\end{document}
